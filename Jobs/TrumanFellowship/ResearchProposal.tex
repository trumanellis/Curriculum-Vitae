\documentclass[letterpaper,12pt]{article}
% \usepackage[margin=1.0in]{geometry}
\title{Research Proposal}
\author{Truman E. Ellis, Pavel Bochev (Sponsor)}
\date{}
\def\etal{{\it et al.~}}

% A research proposal (not to exceed ten pages plus appendices) describing the scientific importance of 
% the proposed work, including considerable specificity as to the research plan and data to be obtained, 
% written for a broad technical audience with field-specific terms explained fully. In the proposal, 
% present the goals of the proposed research and plans to achieve them. 
% Provide a clear objective of the research, the expected results, and the impact that the proposed 
% work may have on Sandia’s Science and Technology capabilities. Include background material,
% such as previous studies done, and motivation for the proposed research. If desired, copies of 
% reprints may be attached to each collated copy of the proposal. State the planned approach in order 
% to clarify the direction of the research. Propose a schedule for the research and explain the 
% importance and benefits to Sandia. No budget is required unless supplementary support is requested 
% as described in the following section.

\begin{document}
\maketitle

% Scientific Importance
% - Previous Studies
% - Motivation for Proposed Research
\section*{Scientific Importance}
\paragraph{DPG -- a robust framework for computational mechanics.}
The discontinuous Petrov-Galerkin (DPG) finite element methodology was first proposed by Demkowicz and Gopalakrishnan \cite{DPG1,DPG2} 
in 2009 and has since gained significant interest from computational mechanics researchers.
The primary strength of the method lies in its suitability to a very broad class of well-posed problems.
DPG offers a fundamental framework for developing robust residual-minimizing finite element methods, even for equations that usually cause problems
for standard techniques, such as convection-dominated diffusion and Stokes flow.
DPG methods have provably optimal convergence rates and often come very close to the best solution possible in the discrete space.

In contrast to most other numerical methods, the discontinuous Petrov-Galerkin finite element method retains exceptional stability on extremely coarse meshes.
DPG is also inherently very adaptive.
It is possible to compute the residual error without knowledge of the exact solution, which can be used to robustly drive adaptivity.
This results in a very automated technology, as the user can initialize a computation on the coarsest mesh which adequately represents the geometry 
then step back and let the program solve and adapt iteratively until it resolves the solution features.

Mesh design for computational mechanics can be a time-consuming and expensive process.
The stability properties and nonlinear convergence of most numerical methods rely on a minimum level of mesh resolution. 
This means that unless the initial computational mesh is fine enough, convergence can not be guaranteed. 
Any meshes below this minimum resolution level are termed to be in the ``pre-asymptotic regime.'' 
This condition implies that meshes need to in some way anticipate the solution before it is known. 
On top of the minimum requirement that the surface meshes must adequately represent the geometry of the problem under consideration, 
resolution requirements on the volume mesh make the computational mechanics practitioner's job significantly more time consuming. 
% This is not to mention that for CFD we also have auxiliary requirements from turbulence models needed to accurately resolve boundary layers. 
Thus, mesh design and simulation become an iterative trial and error process.
A common process is for an engineer to study the problem at hand and attempt to predict regions that require extra resolution. 
Then they will spend hours coaxing the mesher to produce an adequate mesh before importing the mesh into the solver.
Too often, the solver will fail to converge on the given mesh due to some unforeseen mesh inadequacy on part of the solution domain.
The engineer then needs to descend back into the mesher to fix the problem elements.
This process is repeated until the solver is satisfied and convergence can be reached.
This iterative trial and error is undesirable while working on personal computers or modestly sized compute clusters, 
but becomes increasing costly as the problem is scaled up to the tens or hundreds of thousands of processors common in high performance computing environments.



In the context of linear problems, DPG has been successfully applied to a wide range of physical applications.  
Theoretical groundwork for DPG applied to the Poisson equation was laid in \cite{DPGPoisson}.  The time-harmonic Helmholtz equation was the focus of \cite{DPGHelmholtz, Gopalakrishnan2014} and \cite{DPG4}.  Linear elasticity and plate problems were addressed in \cite{BramwellDPG}, \cite{NiemiBramwellDemkowicz10}, and \cite{BramwellDemkowiczQiu10}, and Maxwell was addressed in \cite{DPGCloaking, WohlmuthReport}.  Linear fluid dynamics problems include convection-diffusion \cite{DPG3,DemkowiczHeuer,ChanHeuerThanhDemkowicz2012,Chan2013,EllisLC} and stationary Stokes flow\cite{DPGStokes,EllisLC}.
The robust stability properties of DPG appear to carry over into the nonlinear regime as well; Moro, Peraire and Nguyen observed \cite{MoroNguyenPeraire11, MoroMastersThesis} that on a single element, the HDPG method (referring to DPG applied to a hybridized Discontinuous Galerkin method) converges, whereas HDG on its own does not.  Moreover, HDPG required an order of magnitude less artificial diffusion to converge to a nonlinear solution of the compressible Navier-Stokes.  Chan \etal \cite{Chan2013dpg} showed that for both the viscous Burgers' equation and the compressible Navier Stokes equations at high Mach/Reynolds number, DPG converges on an extremely coarse mesh, allowing the use of an adaptive scheme to capture shock and boundary layer phenomena starting from an initial mesh of $O(1)$ elements.  Roberts reports similar results boasting similar stability for high Reynolds numbers for the incompressible Navier-Stokes equations \cite{NateDissertation}.  
The issue of local conservation within the DPG framework was addressed by Ellis \etal \cite{EllisLC}.

\paragraph{Space-time DPG for transient problems.}
Previous explorations of the DPG method focused exclusively on steady state problems.
The easiest extension of steady DPG to transient problems would be to do an implicit time stepping technique in time 
and use DPG for only the spatial solve at each time step.
We did indeed explore this approach, but it didn't seem to be a natural fit with the adaptive features of DPG.
Clearly the Courant-Friedrichs-Lewy (CFL) condition is not binding since we were interested in implicit time integration schemes, 
but it can be a guiding principle for temporal accuracy in this case.
So if we are interested in temporally accurate solutions, we are limited by the fact that our smallest mesh elements 
(which may be orders of magnitude smaller than the largest elements) are constrained to proceed at a much smaller time step than the mesh as a whole. 
We can either restrict the whole mesh to the smallest time step, or we can attempt some sort of local time stepping.
A space-time DPG formulation presents an attractive choice as we will be able to preserve our natural adaptivity 
from the steady problems while extending it in time.
Thus we achieve an adaptive solution technique for transient problems in a unified framework.
Space-time methods have also been found superior in their satisfaction of geometric conservation laws for problems with moving boundaries \cite{GCL}.
Van der Vegt and van der Ven \cite{vanderVegtEuler} note that that ``The space-time DG method provides optimal efficiency for adapting and deforming the mesh while maintaining a conservative scheme which does not require interpolation of data after mesh refinement or deformation.''
% The obvious downside to such an approach is that for 2D spatial problems, 
% we now have to compute on a three dimensional mesh while a spatially 3D problem becomes four dimensional.

% Research Plan
% - Objective of Research
% - Expected Results
% - Impact on Sandia's Science and Technology capabilities
% - Research Directions
% - Schedule of Research
%  * Improve Scaling
%  * Complex Valued Problems
%  * Collaborate with Other Sandia Researchers
%  * Maxwell
%  * MHD
\section*{Research Plan}
\paragraph{Research directions.}
% - Objective of Research
% - Expected Results
The ultimate objective for this research is to refine and extend my work on space-time DPG for transient fluid dynamics.
My dissertation research will focus on 2D compressible and incompressible flow, 
but I am excited to explore the application of these ideas to problems outside of pure fluid dynamics.
% but from our experience with DPG's stability properties, we expect that application of these ideas to problems outside of fluid dynamics to be straightforward.
In particular, the field of magnetohydrodynamics (MHD) appears to be a promising target for space-time DPG.
Initial investigations applying DPG the Maxwell's equations have been encouraging.
Coupling Maxwell with my work on Navier-Stokes will not be trivial, but we expect this approach to be quite fruitful.
Typically the techniques used to solve fluid dynamics and electromagnetics are very different, but within the DPG framework these can both be solved
with the same technology.
Depending on the level of interest and collaborations I am able to foster at Sandia, 
other research directions could include transient elastodynamics, wave propagation, heat transfer, or turbulence,
but I believe the MHD option holds the most promise.
From my experience at Lawrence Livermore, I understand that multiphase flows are an important aspect of 
simulations at the DOE labs, thus
I would also like to incorporate Ju Liu's recent work (in collaboration with Tom Hughes) on thermodynamically consistent multiphase flow models
into our DPG framework.

% \paragraph{DPG and HPC.}
Many of the features inherent in the DPG method appear promising in the context of high performance computing.
DPG is very compute intensive compared to the associated communication and memory costs.
Most of the work is spent in embarrassingly parallel local solves for the optimal test functions and local stiffness matrix assembly.
Additionally, the stability properties of DPG make high order stability a triviality, and in general, 
high order methods tend to have a more attractive compute/communication profile than low order methods.
In our codes, we use QR factorization for optimal test function solves, but this factorization is recyclable as we essentially have many right hand sides.
The division of degrees of freedom into internal vs element interface unknowns produces a global system which can be statically condensed into 
a solve purely in terms of the element interface degrees of freedom.
In addition to significantly cutting down on the size of the global solve, 
this produces a embarrassingly parallel post-processing solve for the internal degrees of freedom.
This property was one of the motivations behind the development of the hybridized discontinuous Galerkin (HDG) \cite{HDG} method.
No matter what system of equations is being considered, DPG always produces a Hermitian (symmetric if real) 
positive definite stiffness matrix for the global solve.
This property has not really been leveraged in our simulations so far, since we have focused on direct rather than iterative solvers, but we
anticipate it might be an attractive feature in the future.
As compute resources scale up, many more HPC simulations are increasingly becoming coupled in multiphysics simulations.
Since the only requirement for a well-defined discrete DPG method is a well-defined continuous problem, 
it is certainly possible that each different part of the multiphysics simulation could be discretized with DPG -- 
no need to develop many different methods for each part of the simulation.
Thus, another research direction I am interested in pursuing is the development of DPG's potential 
for modern and prospective high performance computing architectures.
% Thus, another promising research direction I am interested in is developing and optimizing space-time DPG for modern and upcoming high performance computing architectures.


\paragraph{Benefits to Sandia.}
% - Impact on Sandia's Science and Technology capabilities
In terms of computational technology, I am developing a new finite element method with significant promise for multiphysics simulations.
In particular, the space-time approach caries substantial benefits over time stepping when dealing with adaptivity and solution features on multiple scales.
The robustness of DPG on coarse meshes with automatic adaptivity promises an easier mesh design process for computational scientists.
Scalability of the method is currently under very active development, but inherent features of DPG and early results are very encouraging.
This research will advance development of the Camellia \cite{CamelliaDPG} DPG library which was initiated at Sandia 
and is built exclusively on top of Trilinos \cite{Trilinos}, another major Sandia effort.
Camellia supports rapid application development and is much simpler to use (interface inspired by the FEniCS project) 
than most general finite element libraries available.
In the next year, Camellia is expected to be released as an open source project, possibly as a module within Trilinos.
% I am an active contributor to the Camellia \cite{CamelliaDPG} DPG library which was started at Sandia and makes heavy use of Trilinos \cite{Trilinos}.
% The primary developer, Nathan Roberts, whom I will be collaborating with on this research, 
% develops Camellia full time at Argonne National Lab.
% In the next year, he plans on releasing Camellia as an open source project, possibly as another module in Trilinos.
% Additionally, I will be making inroads developing a numerical technology which has the potential to 
% contribute to Sandia's greater computational science efforts.
Usage of Sandia's high performance computing systems will be critical to the success of this project, but this project should ultimately
make it easier for other Sandia researchers to leverage those resources to solve their own computational problems,
as DPG and Camellia provide a framework for scientific computation rather than simply a driver to accomplish a single class of problems.

\paragraph{Expected schedule.}
% - Schedule of Research
%  * Improve Scaling
%  * Complex Valued Problems
%  * Collaborate with Other Sandia Researchers
%  * Maxwell
%  * MHD
% 6 months - work on parallel scaling
% 6 months - add support for complex valued problems, Helmholtz
% 6 months - steady state Maxwell
% 6 months - transient Maxwell
% 6 months - transient MHD
% 6 months - explore other transient problems
Predicting the pace of research is obviously an inexact science, but by my rough estimates, my work would progress as follows.
The majority of scientifically interesting electromagnetics problems take place in 3D, so the priority for the first year will 
be improving the performance and scaling of Camellia for 3D problems and exploring iterative solvers and preconditioners.
Scaling and performance improvements are likely to continue through the duration of the research as new bottlenecks and improvements come to light.
I will also be working on adding support for complex valued problems. The Helmholtz equation will serve as a model problem during this development.
The second year will be devoted to solving Maxwell's equations in 3D starting with the steady state version and moving to space-time transient simulations.
I will likely begin initial 2D magnetohydrodynamics simulations at this point.
In the third year, I will begin working on increasingly complex 3D magnetohydrodynamics problems and multiphase flows.
If time permits, I would also like to develop a Python interface to Camellia.
Throughout this process, I intend to build collaborations with Sandia researchers who are already working on these problems so I can 
learn from their expertise and pursue problems in alignment with Sandia's greater goals.
% Throughout this whole process I intend to build collaborations with other Sandia researchers and attempt to guide my simulations in directions of interest 
% to the greater lab.

\paragraph{Expected results.}
The mathematical understanding of the DPG method has matured significantly over the last few years, 
and numerical experiments have confirmed the developed theory.
Based on our previous experience with DPG for fluid flow and new results for DPG for electromagnetism, we are very confident
that magnetohydrodynamics and multiphase flows are a reasonable objective for the Truman Fellowship.
There are some difficult computing challenges to overcome, but they are by no means insurmountable.
By the end of my postdoctoral experience, I believe that we will have a user friendly DPG library that we can present to the greater 
computational community for straightforward computations of realistic scientific and engineering problems on high performance computing systems.
% Magnetohydrodynamics falls into the class of problems that DPG promises stability for, so we don't see any insurmountable obstacles for this 
% Thus, we expect that solving this new class of problems will not pose a problem.
% The challenges then lie on the computing side of things. 
% Making our code efficient and scalable for 3D problems will not be trivial, but we expect it to be doable.
% From several years of experience now working on DPG and Camellia, I do believe that 3D magnetohydrodynamics simulations with multiple materials
% on realistic geometries is a reasonable goal for this postdoc.


\bibliographystyle{elsarticle-num} 
\bibliography{../Papers}
\end{document}
