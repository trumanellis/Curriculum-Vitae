% Cover letter using letter.sty
\documentclass{letter} % Uses 10pt
%Use \documentstyle[newcent]{letter} for New Century Schoolbook postscript font
% the following commands control the margins:
\topmargin=-1in    % Make letterhead start about 1 inch from top of page 
\textheight=8in  % text height can be bigger for a longer letter
\oddsidemargin=0pt % leftmargin is 1 inch
\textwidth=6.5in   % textwidth of 6.5in leaves 1 inch for right margin

\begin{document}

\signature{Truman E. Ellis}           % name for signature 
\longindentation=0pt                       % needed to get closing flush left
\let\raggedleft\raggedright                % needed to get date flush left
 
 
\begin{letter}{
Professor Y. M. Gupta\\
ISP/Applied Sciences Laboratory\\
Washington State University\\
100 Dairy Road, Room 202\\
Pullman, WA 99164-1120\\
}

\begin{flushleft}
{\large\bf Truman E. Ellis}
\end{flushleft}
\medskip\hrule height 1pt
\begin{flushright}
\hfill 2201 Montclaire St. Unit B, Austin, TX 78704 \\
\hfill ellis.truman@gmail.com\\
\hfill 512-814-8304\\
\end{flushright} 
\vfill % forces letterhead to top of page

 
\opening{Dr. Gupta:} 
 
\noindent I just came across your advertised position for Computational Research Scientist at the Applied Sciences Laboratory.
The provided description of ASL was very intriguing as it aligned closely with my ideal combination of private industry and academia.
Admittedly, I don't meet the experience criteria you are looking for (3 years of postdoctoral research and grant history), 
but perhaps there is a postdoc or similar position that I could begin with before transitioning into a more permanent role.
Your research environment sounds very similar to what I experienced through four summers at Lawrence Livermore National Laboratory
where I worked on developing modern compressible flow solvers for simulating materials under extreme conditions.
I should be completing my Ph.D. at the Institute for Computational Engineering and Science at the University of Texas next summer.
ICES is an interdisciplinary research unit that teaches the mathematics, physical modeling, and computational know-how behind scientific computing.
As such, I have a strong background in computational mechanics as a whole with a special emphasis on computational fluid dynamics.
My Ph.D. experience has rounded out my knowledge of the mathematics behind numerical algorithms and the various techniques needed to develop
stable methods for various types of physical phenomena.
While I do not currently have grant writing experience, I believe this is something I could grow to excel at given some mentorship.
It is important to me that my research brings ultimate benefit to society, and I pride myself on being able to communicate 
that benefit clearly and persuasively.

My current research has been focused on developing the discontinuous Petrov-Galerkin (DPG) finite element method for fluid dynamics applications.
DPG offers a fundamental framework for developing robust residual-minimizing finite element methods, even for equations that usually cause problems
for standard techniques, such as convection-dominated diffusion and Stokes flow.
The strength of the technique lies in the straightforward application to any well-posed PDE due to DPG's superior stability properties.
I recently published a paper on my work developing a locally conservative DPG formulation and submitted a paper detailing a space-time formulation
for transient problems.
Given complete freedom, I would pursue the extension of my current work to magnetohydrodynamics and multiphase flows 
but I am amenable to other pursuits if funding favors other directions.
I have had some excellent opportunities for collaborations both at LLNL and ICES. 
Should you wish to speak with my advisors, at LLNL they were Tzanio Kolev (kolev1@llnl.gov, 925-423-9797) and Rob Rieben (rieben1@llnl.gov, 925-422-3783).
My current advisor is Leszek Demkowicz (leszek@ices.utexas.edu, 512-471-4199).

% I recognize that I don't fit the advertised requirements but I would appreciate an opportunity to discuss alternative placements at ASL or the Institute for Shock Physics.
While I am pursuing various other postdoctoral positions and industry jobs, the environment and location of ASL appeal 
more than most of the alternatives.
I am in candidacy and working on research full time at the moment, so my schedule is fairly open should you wish to chat.
The prospect of working at the Applied Sciences Laboratory is very exciting to me and I would welcome a chance to discuss a possible role there.
Thank you for your time and consideration.
% If the lack of postdoctoral experience is a deal breaker, then perhaps I might revisit this option a few years down the road if I pursue a postdoc at another location.
% \noindent PARAGRAPH ONE: State reason for letter, name the position or type 
% of work you are applying for and identify source from  which  you 
% learned   of   the  opening.  (i.e.  Career  Development  Center, 
% newspaper, employment service, personal contact). 
 
% \noindent PARAGRAPH  TWO:  Indicate why you are interested in the position, 
% the company, its products, services - above all, stress what  you 
% can  do  for  the employer. If you are a recent graduate, explain 
% how your academic background makes you a qualified candidate  for 
% the  position.  If  you have practical work experience, point out 
% specific achievements or unique qualifications. Try not to repeat 
% the  same  information  the reader will find in the resume. Refer 
% the reader to the enclosed resume or application which summarizes 
% your  qualifications,  training,  and experiences. The purpose of 
% this section is to strengthen your resume  by  providing  details 
% which bring your experiences to life. 
 
% \noindent PARAGRAPH THREE: Request a personal interview and  indicate  your 
% flexibility as to the time and place. Repeat your phone number in 
% the letter and offer assistance to help in a speedy response. For 
% example,  state that you will be in the city where the company is 
% located on a certain date and would like to set up an  interview. 
% Or,  state  that  you  will  call  on a certain date to set up an 
% interview. End the letter by thanking  the  employer  for  taking 
% time to consider your credentials. 
 
\closing{Sincerely,} 
 

 
\encl{Curriculum Vitae}  				% Enclosures

\end{letter}
 

\end{document}






