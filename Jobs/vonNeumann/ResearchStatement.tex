\documentclass[letterpaper,12pt]{article}
\usepackage[margin=1.0in]{geometry}
\title{Research Statement}
\author{Truman E. Ellis}
\date{}
\def\etal{{\it et al.~}}

\begin{document}
\maketitle

% Para 1: A brief paragraph sketching the overarching thematics and topic of your research, situating it disciplinarily.

% Para 2: A summary of the dissertation research. This may replicate to some extent the paragraph on the dissertation in the cover letter, but it must have more detail about the methods, the theoretical foundations, and most of all, the core arguments.  Here, give a chapter summary, approximately one sentence per chapter.
% \section*{Summary of Current Research}

% Para 3: A brief description of the contribution of the dissertation research to your field or fields, and a summary of publications associated with the dissertation research, including a plan for the book, if you are in a book field.

% [Addendum: Other ongoing research can be described here in more paragraphs in the event that you are going for a 2 page document]

% Para 4: A summary of the next research project, providing a topic, methods, a theoretical orientation, and brief statement of contribution to your field or fields. Mention publications, conference talks, or grants related to the new project.

% Para 5: A brief summary of the wider impact of your research agenda(s) writ large—what do they “tell us” that is valuable and important, both for a discipline but also for a wider scholarly community, and in some cases, for humanity in general.

\section*{Introduction}
My current research involves the development of the discontinuous Petrov-Galerkin (DPG) finite element method for space-time 
simulations of compressible and incompressible fluid dynamics.
DPG represents the generalization and systemization of several ideas in stabilized finite element methods and provides a rich methodology
for solving partial differential equations.
% The DPG framework supports any well posed problem, promising successful application to multiphysics simulations.
% Exceptional stability properties support a wide class of different discretizations and pave the way toward automated scientific computing.
% The strengths of the DPG framework lie in its applications to multiphysics simulations, superior stability properties 
% which pave the way toward automated scientific computing.

\section*{Scientific Importance}
\paragraph{DPG -- a robust framework for computational mechanics.}
The primary strength of the method lies in its suitability to any well-posed variational problem.
% very broad class of well-posed problems.
DPG offers a fundamental framework for developing robust residual-minimizing finite element methods, even for equations that usually cause problems
for standard techniques, such as convection-dominated diffusion and Stokes flow.
DPG methods have provably optimal convergence rates and often come very close to the best solution possible in the discrete space.

In contrast to most other numerical methods, the discontinuous Petrov-Galerkin finite element method retains exceptional stability on extremely coarse meshes.
DPG is also inherently very adaptive.
It is possible to compute the residual error without knowledge of the exact solution, which can be used to robustly drive adaptivity.
This results in a very automated technology, as the user can initialize a computation on the coarsest mesh which adequately represents the geometry 
then step back and let the program solve and adapt iteratively until it resolves the solution features.

DPG has been successfully applied to a wide range of physical applications.  
Theoretical groundwork for DPG applied to the Poisson equation was laid in \cite{DPGPoisson}.  The time-harmonic Helmholtz equation was the focus of \cite{DPGHelmholtz, Gopalakrishnan2014} and \cite{DPG4}.  Linear elasticity and plate problems were addressed in \cite{BramwellDPG}, \cite{NiemiBramwellDemkowicz10}, and \cite{BramwellDemkowiczQiu10}, and Maxwell was addressed in \cite{DPGCloaking, WohlmuthReport}.  Linear fluid dynamics problems include convection-diffusion \cite{DPG3,DemkowiczHeuer,ChanHeuerThanhDemkowicz2012,Chan2013,EllisLC} and stationary Stokes flow\cite{DPGStokes,EllisLC}.
The robust stability properties of DPG appear to carry over into the nonlinear regime as well; Moro, Peraire and Nguyen observed \cite{MoroNguyenPeraire11, MoroMastersThesis} that on a single element, the HDPG method (referring to DPG applied to a hybridized Discontinuous Galerkin method) converges, whereas HDG on its own does not.  Moreover, HDPG required an order of magnitude less artificial diffusion to converge to a nonlinear solution of the compressible Navier-Stokes equations.  Chan \etal \cite{Chan2013dpg} showed that for both the viscous Burgers' equation and the compressible Navier-Stokes equations at high Mach/Reynolds number, DPG converges on an extremely coarse mesh, allowing the use of an adaptive scheme to capture shock and boundary layer phenomena starting from an initial mesh of $O(1)$ elements.  Roberts reports similar results boasting similar stability for high Reynolds numbers for the incompressible Navier-Stokes equations \cite{NateDissertation}.  
% The issue of local conservation within the DPG framework was addressed by Ellis \etal \cite{EllisLC}.
% I recently addressed the issue of local conservation within the DPG framework in \cite{EllisLC}.

\section*{Current Research}
\paragraph{DPG for fluid dynamics applications.}
Local conservation is considered by many fluid dynamics practitioners to be an extremely important feature in a CFD code. 
Thus, my initial work on DPG included the development of a variant formulation that explicitly enforced conservation element-by-element 
via Lagrange multipliers \cite{EllisLC}.

Past explorations of the DPG method focused exclusively on steady state problems.
% High Reynolds number flows (and many other realistic engineering simulations) often require highly adaptive meshes for efficient computation,
Most simulations of realistic engineering applications require highly adaptive meshes for efficient computation,
but traditional time stepping algorithms are horribly inefficient in such cases as the entire simulation needs to move forward 
at the pace of the most restrictive element, which may be orders of magnitude smaller than other elements in the domain.
% Traditional time stepping algorithms do not lend themselves very well to highly adaptive meshes, such as those produced by adaptively solving 
% for a singularly perturbed problem as element sizes can vary by several orders of magnitude.
% The highly adaptive nature of DPG simulations does not lend itself very well to traditional time stepping algorithms,
% as elements may vary in size over several orders of magnitude.
% Traditional time stepping algorithms can be very inefficient when working with highly adaptive meshes 
% as elements may vary in size of several orders of magnitude.
% My current work extends the applications of DPG to the transient regime by using a space-time formulation.
% The easiest extension of steady DPG to transient problems would be to do an implicit time stepping technique in time 
% and use DPG for only the spatial solve at each time step.
% We did indeed explore this approach, but it didn't seem to be a natural fit with the adaptive features of DPG.
% Clearly the Courant-Friedrichs-Lewy (CFL) condition is not binding since we were interested in implicit time integration schemes, 
% but it can be a guiding principle for temporal accuracy in this case.
% So if we are interested in temporally accurate solutions, we are limited by the fact that our smallest mesh elements 
% (which may be orders of magnitude smaller than the largest elements) are constrained to proceed at a much smaller time step than the mesh as a whole. 
% We can either restrict the whole mesh to the smallest time step, or we can attempt some sort of local time stepping.
A space-time DPG formulation presents an attractive alternative as we will be able to preserve our natural stability and adaptivity 
from the steady problem while extending it in time.
Thus we achieve an adaptive solution technique for transient problems in a unified framework.
% Space-time methods have also been found superior in their satisfaction of geometric conservation laws for problems with moving boundaries \cite{GCL}.
% Van der Vegt and van der Ven \cite{vanderVegtEuler} note that that ``The space-time DG method provides optimal efficiency for adapting and deforming the mesh while maintaining a conservative scheme which does not require interpolation of data after mesh refinement or deformation.''

More recently, I've been using the framework we've developed to study a classical open question in compressible flow problems: 
whether primitive, conservation, or entropy variables are ideal for fluid simulations.
A related capability that we have been developing is the ability to use entropy to develop a physically relevant measure of our residual.
% The stability properties of DPG provide an ideal framework within which to compare Navier-Stokes simulations with 
% primitive, conservation, or entropy variables.

\section*{Future Research}
\paragraph{DPG for multiphysics and magnetohydrodynamics.}
The field of magnetohydrodynamics (MHD) appears to be a promising target for space-time DPG.
Initial investigations applying DPG the Maxwell's equations have been encouraging.
Coupling Maxwell with my work on Navier-Stokes will not be trivial, but we expect this approach to be quite fruitful.
Typically the techniques used to solve fluid dynamics and electromagnetics are very different, but within the DPG framework these can both be solved
with the same technology.
Depending on the level of interest and collaborations I am able to foster at Sandia, 
other research directions could include transient elastodynamics, wave propagation, heat transfer, or turbulence,
but I believe the MHD option holds the most promise.
From my experience at Lawrence Livermore, I understand that multiphase flows are an important aspect of 
simulations at the DOE labs, thus
I would also like to incorporate Ju Liu's recent work (in collaboration with Tom Hughes) on thermodynamically consistent multiphase flow models
into our DPG framework.

I would also like to further develop the marriage of DPG and high performance computing.
Many of the features inherent to the DPG method appear promising in the context of massively parallel simulations.
DPG is very compute intensive compared to the associated communication and memory costs.
Most of the work is spent in embarrassingly parallel local solves for the optimal test functions and local stiffness matrix assembly.
Additionally, the stability properties of DPG make high order stability a triviality, and in general, 
high order methods tend to have a more attractive compute/communication profile than low order methods.
In our codes, we use QR factorization for optimal test function solves, but this factorization is recyclable as we essentially have many right hand sides.
The division of degrees of freedom into internal vs element interface unknowns produces a global system which can be statically condensed into 
a solve purely in terms of the element interface degrees of freedom.
In addition to significantly cutting down on the size of the global solve, 
this produces a embarrassingly parallel post-processing solve for the internal degrees of freedom.
This property was one of the motivations behind the development of the hybridized discontinuous Galerkin (HDG) \cite{HDG} method.
No matter what system of equations is being considered, DPG always produces a Hermitian 
positive definite stiffness matrix for the global solve.
This property has not really been leveraged in our simulations so far, since we have focused on direct rather than iterative solvers, but we
anticipate it might be an attractive feature in the future.
% As compute resources scale up, HPC simulations are increasingly becoming coupled in multiphysics simulations.
Multiphysics simulations are becoming increasingly accessible and desirable as compute resources scale up.
Since the only requirement for a well-defined discrete DPG method is a well-defined continuous problem, 
it is certainly possible that each part of a multiphysics simulation could be discretized with DPG -- 
no need to develop many different methods for each part of the simulation.
% Thus, another research direction I am interested in pursuing is the development of DPG's potential 
% for modern and prospective high performance computing architectures.

\section*{Wider Impact}
\paragraph{Benefits to Sandia.}
In terms of computational technology, I am developing a new finite element method with significant promise for multiphysics simulations.
In particular, the space-time approach caries substantial benefits over time stepping when dealing with adaptivity and solution features on multiple scales.
The robustness of DPG on coarse meshes with automatic adaptivity promises an easier mesh design process for computational scientists.
Scalability of the method is currently under very active development, but inherent features of DPG and early results are very encouraging.
This research will advance development of the Camellia \cite{CamelliaDPG} DPG library which was initiated at Sandia 
and is built exclusively on top of Trilinos \cite{Trilinos}. %, another major Sandia effort.
Camellia supports rapid application development and is much simpler to use
than most general finite element libraries available, with an interface inspired by the FEniCS project.
In the next year, Camellia is expected to be released as an open source project, possibly as a module within Trilinos.
% Usage of Sandia's high performance computing systems will be critical to the success of this project, but 
This project should ultimately
make it easier for other Sandia researchers to leverage HPC resources to solve their own computational problems,
as DPG and Camellia provide a framework for scientific computation rather than simply a driver to solve a single class of problems.

Based on our previous experience with DPG for fluid flow and new results for DPG for electromagnetism, we are very confident
that magnetohydrodynamics and multiphase flows are a reasonable objective for the von Neumann Fellowship.
There are some difficult computing challenges to overcome, but they are by no means insurmountable.
By the end of my postdoctoral experience, I believe that we will have a user friendly DPG library that we can present to the greater 
computational community for straightforward computations of realistic scientific and engineering problems on high performance computing systems.

\bibliographystyle{elsarticle-num} 
\bibliography{../Papers}
\end{document}
