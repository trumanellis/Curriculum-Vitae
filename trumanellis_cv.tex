%%%%%%%%%%%%%%%%%%%%%%%%%%%%%%%%%%%%%%%%%%%%%%%%%%%%%%%%%%%%%%%%%%%%%%%%
%%%%%%%%%%%%%%%%%%%%%% Simple LaTeX CV Template %%%%%%%%%%%%%%%%%%%%%%%%
%%%%%%%%%%%%%%%%%%%%%%%%%%%%%%%%%%%%%%%%%%%%%%%%%%%%%%%%%%%%%%%%%%%%%%%%

%%%%%%%%%%%%%%%%%%%%%%%%%%%%%%%%%%%%%%%%%%%%%%%%%%%%%%%%%%%%%%%%%%%%%%%%
%% NOTE: If you find that it says                                     %%
%%                                                                    %%
%%                           1 of ??                                  %%
%%                                                                    %%
%% at the bottom of your first page, this means that the AUX file     %%
%% was not available when you ran LaTeX on this source. Simply RERUN  %%
%% LaTeX to get the ``??'' replaced with the number of the last page  %%
%% of the document. The AUX file will be generated on the first run   %%
%% of LaTeX and used on the second run to fill in all of the          %%
%% references.                                                        %%
%%%%%%%%%%%%%%%%%%%%%%%%%%%%%%%%%%%%%%%%%%%%%%%%%%%%%%%%%%%%%%%%%%%%%%%%

%%%%%%%%%%%%%%%%%%%%%%%%%%%% Document Setup %%%%%%%%%%%%%%%%%%%%%%%%%%%%

% Don't like 10pt? Try 11pt or 12pt
\documentclass[10pt]{article}

% This is a helpful package that puts math inside length specifications
\usepackage{calc}
\usepackage{bibentry}

% Turn of hyphenation
\hyphenpenalty=5000

% Simpler bibsection for CV sections
% (thanks to natbib for inspiration)
\makeatletter
\newlength{\bibhang}
\setlength{\bibhang}{1em}
\newlength{\bibsep}
 {\@listi \global\bibsep\itemsep \global\advance\bibsep by\parsep}
\newenvironment{bibsection}%
        {\vspace{-\baselineskip}\begin{list}{}{%
       \setlength{\leftmargin}{\bibhang}%
       \setlength{\itemindent}{-\leftmargin}%
       \setlength{\itemsep}{\bibsep}%
       \setlength{\parsep}{\z@}%
        \setlength{\partopsep}{0pt}%
        \setlength{\topsep}{0pt}}}
        {\end{list}\vspace{-.6\baselineskip}}
\makeatother

% Layout: Puts the section titles on left side of page
\reversemarginpar

%
%         PAPER SIZE, PAGE NUMBER, AND DOCUMENT LAYOUT NOTES:
%
% The next \usepackage line changes the layout for CV style section
% headings as marginal notes. It also sets up the paper size as either
% letter or A4. By default, letter was used. If A4 paper is desired,
% comment out the letterpaper lines and uncomment the a4paper lines.
%
% As you can see, the margin widths and section title widths can be
% easily adjusted.
%
% ALSO: Notice that the includefoot option can be commented OUT in order
% to put the PAGE NUMBER *IN* the bottom margin. This will make the
% effective text area larger.
%
% IF YOU WISH TO REMOVE THE ``of LASTPAGE'' next to each page number,
% see the note about the +LP and -LP lines below. Comment out the +LP
% and uncomment the -LP.
%
% IF YOU WISH TO REMOVE PAGE NUMBERS, be sure that the includefoot line
% is uncommented and ALSO uncomment the \pagestyle{empty} a few lines
% below.
%

%% Use these lines for letter-sized paper
\usepackage[paper=letterpaper,
            %includefoot, % Uncomment to put page number above margin
            marginparwidth=1.2in,     % Length of section titles
            marginparsep=.05in,       % Space between titles and text
            margin=1in,               % 1 inch margins
            includemp]{geometry}

%% Use these lines for A4-sized paper
%\usepackage[paper=a4paper,
%            %includefoot, % Uncomment to put page number above margin
%            marginparwidth=30.5mm,    % Length of section titles
%            marginparsep=1.5mm,       % Space between titles and text
%            margin=25mm,              % 25mm margins
%            includemp]{geometry}

%% More layout: Get rid of indenting throughout entire document
\setlength{\parindent}{0in}

%% This gives us fun enumeration environments. compactitem will be nice.
\usepackage{paralist}

%% Reference the last page in the page number
%
% NOTE: comment the +LP line and uncomment the -LP line to have page
%       numbers without the ``of ##'' last page reference)
%
% NOTE: uncomment the \pagestyle{empty} line to get rid of all page
%       numbers (make sure includefoot is commented out above)
%
\usepackage{fancyhdr,lastpage}
\pagestyle{fancy}
%\pagestyle{empty}      % Uncomment this to get rid of page numbers
\fancyhf{}\renewcommand{\headrulewidth}{0pt}
\fancyfootoffset{\marginparsep+\marginparwidth}
\newlength{\footpageshift}
\setlength{\footpageshift}
          {0.5\textwidth+0.5\marginparsep+0.5\marginparwidth-2in}
\lfoot{\hspace{\footpageshift}%
       \parbox{4in}{\, \hfill %
                    \arabic{page} of \protect\pageref*{LastPage} % +LP
%                    \arabic{page}                               % -LP
                    \hfill \,}}

% Finally, give us PDF bookmarks
\usepackage{color,hyperref}
\definecolor{darkblue}{rgb}{0.0,0.0,0.3}
\hypersetup{colorlinks,breaklinks,
            linkcolor=darkblue,urlcolor=darkblue,
            anchorcolor=darkblue,citecolor=darkblue}

%%%%%%%%%%%%%%%%%%%%%%%% End Document Setup %%%%%%%%%%%%%%%%%%%%%%%%%%%%


%%%%%%%%%%%%%%%%%%%%%%%%%%% Helper Commands %%%%%%%%%%%%%%%%%%%%%%%%%%%%

% The title (name) with a horizontal rule under it
% (optional argument typesets an object right-justified across from name
%  as well)
%
% Usage: \makeheading{name}
%        OR
%        \makeheading[right_object]{name}
%
% Place at top of document. It should be the first thing.
% If ``right_object'' is provided in the square-braced optional
% argument, it will be right justified on the same line as ``name'' at
% the top of the CV. For example:
%
%       \makeheading[\emph{Curriculum vitae}]{Your Name}
%
% will put an emphasized ``Curriculum vitae'' at the top of the document
% as a title. Likewise, a picture could be included:
%
%   \makeheading[\includegraphics[height=1.5in]{my_picutre}]{Your Name}
%
% the picture will be flush right across from the name.
\newcommand{\makeheading}[2][]%
        {\hspace*{-\marginparsep minus \marginparwidth}%
         \begin{minipage}[t]{\textwidth+\marginparwidth+\marginparsep}%
             {\large \bfseries #2 \hfill #1}\\[-0.15\baselineskip]%
                 \rule{\columnwidth}{1pt}%
         \end{minipage}}

% The section headings
%
% Usage: \section{section name}
%
% Follow this section IMMEDIATELY with the first line of the section
% text. Do not put whitespace in between. That is, do this:
%
%       \section{My Information}
%       Here is my information.
%
% and NOT this:
%
%       \section{My Information}
%
%       Here is my information.
%
% Otherwise the top of the section header will not line up with the top
% of the section. Of course, using a single comment character (%) on
% empty lines allows for the function of the first example with the
% readability of the second example.
\renewcommand{\section}[2]%
        {\pagebreak[3]\vspace{1.3\baselineskip}%
         \phantomsection\addcontentsline{toc}{section}{#1}%
         \hspace{0in}%
         \marginpar{
         \raggedright \scshape #1}#2}

% An itemize-style list with lots of space between items
\newenvironment{outerlist}[1][\enskip\textbullet]%
        {\begin{itemize}[#1]}{\end{itemize}%
         \vspace{-.6\baselineskip}}

% An environment IDENTICAL to outerlist that has better pre-list spacing
% when used as the first thing in a \section
\newenvironment{lonelist}[1][\enskip\textbullet]%
        {\vspace{-\baselineskip}\begin{list}{#1}{%
        \setlength{\partopsep}{0pt}%
        \setlength{\topsep}{0pt}}}
        {\end{list}\vspace{-.6\baselineskip}}

% An itemize-style list with little space between items
\newenvironment{innerlist}[1][\enskip\textbullet]%
        {\begin{compactitem}[#1]}{\end{compactitem}}

% An environment IDENTICAL to innerlist that has better pre-list spacing
% when used as the first thing in a \section
\newenvironment{loneinnerlist}[1][\enskip\textbullet]%
        {\vspace{-\baselineskip}\begin{compactitem}[#1]}
        {\end{compactitem}\vspace{-.6\baselineskip}}

% To add some paragraph space between lines.
% This also tells LaTeX to preferably break a page on one of these gaps
% if there is a needed pagebreak nearby.
\newcommand{\blankline}{\quad\pagebreak[3]}
\newcommand{\halfblankline}{\quad\vspace{-0.5\baselineskip}\pagebreak[3]}

% Uses hyperref to link DOI
\newcommand\doilink[1]{\href{http://dx.doi.org/#1}{#1}}
\newcommand\doi[1]{doi:\doilink{#1}}

% For \url{SOME_URL}, links SOME_URL to the url SOME_URL
\providecommand*\url[1]{\href{#1}{#1}}
% Same as above, but pretty-prints SOME_URL in teletype fixed-width font
\renewcommand*\url[1]{\href{#1}{\texttt{#1}}}

% For \email{ADDRESS}, links ADDRESS to the url mailto:ADDRESS
\providecommand*\email[1]{\href{mailto:#1}{#1}}
% Same as above, but pretty-prints ADDRESS in teletype fixed-width font
%\renewcommand*\email[1]{\href{mailto:#1}{\texttt{#1}}}

%\providecommand\BibTeX{{\rm B\kern-.05em{\sc i\kern-.025em b}\kern-.08em
%    T\kern-.1667em\lower.7ex\hbox{E}\kern-.125emX}}
%\providecommand\BibTeX{{\rm B\kern-.05em{\sc i\kern-.025em b}\kern-.08em
%    \TeX}}
\providecommand\BibTeX{{B\kern-.05em{\sc i\kern-.025em b}\kern-.08em
    \TeX}}
\providecommand\Matlab{\textsc{Matlab}}

%%%%%%%%%%%%%%%%%%%%%%%% End Helper Commands %%%%%%%%%%%%%%%%%%%%%%%%%%%

%%%%%%%%%%%%%%%%%%%%%%%%% Begin CV Document %%%%%%%%%%%%%%%%%%%%%%%%%%%%

\begin{document}
\makeheading{Truman E.~Ellis}

\section{Contact Information}
%
% NOTE: Mind where the & separators and \\ breaks are in the following
%       table.
%
% ALSO: \rcollength is the width of the right column of the table
%       (adjust it to your liking; default is 1.85in).
%
\newlength{\rcollength}\setlength{\rcollength}{2.00in}%
%
\begin{tabular}[t]{@{}p{\textwidth-\rcollength}p{\rcollength}}
\href{http://www.ices.utexas.edu/}%
     {Institute for Computational Engineering and Sciences} & \\
\href{http://www.utexas.edu/}{The University of Texas at Austin}
                           & \textit{Phone:} 512-814-8304 \\
201 East 24th St, Stop C0200 & \textit{Email:} \email{ellis.truman@gmail.com}\\
Austin, TX 78712-1229          & \textit{Web:}
\href{http://www.trumanellis.com/}{www.trumanellis.com}\\
\end{tabular}

\section{Summary of Qualifications}
%
% Computational engineer with a solid grasp of the Navier-Stokes, Reynolds
% Averaged Navier-Stokes, and Euler equations, and practiced with the numerical
% methods used to solve them.  Experience running commercial CFD solvers as well
% as writing my own. Expertise using Linux on high performance computing
% systems. Well developed programming and development skills. Comfortable with
% both spoken and written communication and interpersonal skills with experience
% working in a team environment as well as individually.

% Computational scientist with a background in aerospace engineering and an emphasis on fluid dynamics.
% Experience running commercial CFD solvers as well
% as writing my own. Expertise using Linux on high performance computing
% systems. Well developed programming and development skills. Comfortable with
% both spoken and written communication and interpersonal skills with experience
% working in a team environment as well as individually.

Computational scientist with a background in aerospace engineering and an emphasis on fluid dynamics.
Moderate exposure to computational solid mechanics, wave propagation, electrodynamics, and heat transfer.
Experience running commercial CFD solvers as well as developing research codes. 
Expertise using Linux on high performance computing systems. 
Well developed programming and development skills with an affinity for clean, elegant solutions. 
Comfortable with both spoken and written communication and interpersonal skills with experience
working in a team environment as well as individually.

\section{Research Interests}
%
Computational fluid dynamics, turbulence modeling,
finite element methods, discontinuous Petrov-Galerkin, Lagrangian hydrocodes,
computational plasma dynamics, magnetohydrodynamics, computational mechanics

\section{Education}
%
\href{http://www.utexas.edu/}{\textbf{The University of Texas}},
Austin
\begin{outerlist}

\item[] Ph.D.,
        \href{http://www.ices.utexas.edu/}
             {Computational Science Engineering and Mathematics}, expected May 2015
        \begin{innerlist}
        \item Thesis Topic: \emph{Space-time Discontinuous Petrov-Galerkin Finite Elements for Transient Computational Fluid Dynamics}
        \item Advisors:
              \href{http://users.ices.utexas.edu/~leszek/}
                   {Leszek~Demkowicz},
              \href{http://www.me.utexas.edu/directory/faculty/moser/robert/131/}
                   {Robert~Moser}
        \end{innerlist}

\end{outerlist}
\bigskip

\href{http://www.calpoly.edu/}{\textbf{California Polytechnic State University}},
San Luis Obispo
\begin{outerlist}
\item[] M.S.,
        \href{http://www.aero.calpoly.edu/}
             {Aerospace Engineering}, June 2010
        \begin{innerlist}
        \item Thesis Topic: \emph{High Order Finite Elements for Lagrangian
              Computational Fluid Dynamics}
        \item Advisors: \href{http://people.llnl.gov/kolev1}{Tzanio~Kolev}, Robert~Rieben,
              Faysal~Kolkailah
        \item \emph{Summa cum Laude}, With Highest Honors in Engineering
        \end{innerlist}

\item[] B.S.,
        \href{http://www.aero.calpoly.edu/}
             {Aerospace Engineering}, June 2010
        \begin{innerlist}
        \item Aeronautics specialization 
        \item \emph{Summa cum Laude}, With Highest Honors in Engineering
        \end{innerlist}

\end{outerlist}

\section{Professional Experience}
%
\textbf{Graduate Research Assistant} \hfill {2010 to present}
\begin{innerlist}

\item[] \href{http://ices.utexas.edu/}{Institute for Computational Engineering
and Sciences},\\
        \href{http://www.utexas.edu/}{University of Texas at Austin}
\begin{innerlist}
\item Developing the discontinuous Petrov-Galerkin finite element method for
fluid flow applications.
\item Actively contributing to
\href{https://github.com/CamelliaDPG/Camellia}{Camellia}, a parallel C++ library built on Trilinos for
rapid development of DPG problem formulations.
% \item Added an interface to VTK output for visualization via Paraview.
\item Wrote bridge code to enable output in VTK and HDF5 formats.
\item Added support for space-time parabolic problems.
\item Implemented an exactly conservative formulation of DPG through Lagrange
multipliers.
\item Contributed to open source
\href{http://libmesh.sourceforge.net/}{libMesh} finite element library.
\end{innerlist}
\end{innerlist}

\bigskip
\pagebreak

\textbf{Graduate Student Researcher} \hfill {2008 to 2010, 2013}
\begin{innerlist}

\item[] \href{http://iscr.llnl.gov/}{Institute for Scientific Computing Research},\\
        \href{http://www.llnl.gov/}{Lawrence Livermore National Laboratory}
\begin{innerlist}
\item Worked in a small research group developing advanced finite element discretization methods for Lagrangian hydrodynamics.
\item Goal was to improve the current staggered grid hydro algorithms in multi-material Arbitrary Lagrangian Eulerian codes.
\item Wrote a prototype code in Matlab to explore the benefits of using high order finite element pairs.
\item Extended \href{https://computation.llnl.gov/casc/blast/blast.html}{Blast}, the next iteration object oriented C++ code to axisymmetric problems
\item Implemented a smoothness indicator to isolate artificial viscosity to shocked and underresolved flow regions.
\item Developed a Python-scriptable 2D plotting tool to interface with the research code
\item Contributed to open source \href{http://code.google.com/p/mfem/}{MFEM} finite element library
% \item Research presented at the 2009 international conference on  Numerical Methods for Multi-Material Fluids and Structures in Pavia, Italy
\end{innerlist}
\end{innerlist}

\bigskip

\textbf{Undergraduate Student Researcher} \hfill {Summer 2007}
\begin{innerlist}

\item[] {Research Experience for Undergraduates},
        \href{http://www.ae.uiuc.edu/}{Aerospace Engineering},\\
        \href{http://www.uiuc.edu/}{University of Illinois at Urbana-Champaign}
\begin{innerlist}
\item \emph{Compressible Flows in Geological Applications} - Designed a series of experiments and set up a lab to study the Mount St. Helens lateral blast
\end{innerlist}
\end{innerlist}

\section{Refereed\\Journal\\Publications}
\begin{bibsection}
    \item Truman Ellis, Leszek Demkowicz, Jesse Chan, and Robert Moser (2014),\\
          Space-Time DPG: Designing a Method for Massively Parallel CFD.\\
          \emph{Computers \& Fluids} (submitted)
    \item Truman Ellis, and Leszek Demkowicz (2014),\\
          Locally Conservative Discontinuous Petrov-Galerkin Finite Elements for Fluid Problems.\\
          \emph{Computers \& Mathematics with Applications} 
          \doi{10.1016/j.camwa.2014.07.005}
    \item Veselin Dobrev, Truman Ellis, Tzanio Kolev and Robert Rieben (2011),\\
          Curvilinear Finite Elements for Lagrangian Hydrodynamics.\\
          \emph{International Journal for Numerical Methods in Fluids},
          \doi{10.1002/fld.2366}
    \item Veselin Dobrev, Truman Ellis, Tzanio Kolev and Robert Rieben (2012),\\
          High-order Curvilinear Finite Elements for Axisymmetric Lagrangian
          Hydrodynamics.\\
          \emph{Computers \& Fluids},
          \doi{10.1016/j.compfluid.2012.06.004}
\end{bibsection}

% \section{Submitted\\Journal\\Publications}
% \begin{bibsection}
%     \item Dobrev, V. A., Ellis, T. E., Kolev, T. V. and Rieben, R. N. (2011),
%           Curvilinear finite elements for Lagrangian hydrodynamics.
%           \emph{International Journal for Numerical Methods in Fluids},
%           65: 1295–1310. \doi{10.1002/fld.2366}
% \end{bibsection}

% Add a little space to nudge next ``Conference Publications'' marginpar
% down to make room for tall ``Submitted Journal Publications''
% marginpar. If there are enough submitted journal publications, this
% space will not be needed (and should be removed).
% \vspace{0.1in}

% \section{Papers in Preparation} \begin{bibsection}
%     \item Pavlic, T.P., K.M.~Passino. Distributed optimization under
%         constraints: Pareto-optimal intelligent lighting.
%
%     \item Pavlic, T.P. The ideal free distribution as degenerate form of
%         nutrient-constrained optimization.
% \end{bibsection}

\section{Software Skills}
%
Computer Programming:
%
\begin{innerlist}
    \item C$+$$+$, Python, Lua, \Matlab, Mathematica, and others
\end{innerlist}

\halfblankline

CFD / Engineering Software:
%
\begin{innerlist}
    \item Fluent, Gambit, SolidWorks, Pro/ENGINEER, and others
\end{innerlist}

\halfblankline

% Version Control and Software Configuration Management:
% %
% \begin{innerlist}
%     \item Git, SVN
% \end{innerlist}

% \halfblankline

% Productivity Applications:
% %
% \begin{innerlist}
%     \item \LaTeX{}, Vim, OpenOffice/LibreOffice/Microsoft Office, and others
% \end{innerlist}

% \halfblankline

% Operating Systems:
% %
% \begin{innerlist}
%     \item Linux, Apple OS X, Microsoft Windows family
% \end{innerlist}

\section{Awards}
%
\begin{innerlist}
\item Computational Applied Math Fellow -- University of Texas
\item Graduated \emph{Summa cum Laude} -- Cal Poly
\item President's Honors List -- Cal Poly 2005 - 2007
\item Dean's List -- Cal Poly 2005 - 2008
\item Litton Industries in Engineering Scholarship -- Cal Poly 2007 - 2008
\item Accenture Outstanding AERO Award -- Cal Poly 2007
\item Reinhold Aerospace Engineering Scholarship -- Cal Poly 2007
% \item Dean's List -- Ventura College 2002 - 2005
% \item Howe Heywood Mathematics Prize -- Ventura College 2005
% \item James and Ida Iliff Memorial Scholarship -- Ventura College 2005
% \item Alexis Dember Scholarship -- Ventura College 2005
% \item Alpha Gamma Sigma Scholastic and Service Award -- Ventura College 2003
\end{innerlist}

% \section{Professional Memberships}
% %
% \begin{innerlist}
% \item President -- Sigma Gamma Tau Aerospace Honor Society -- Cal Poly 2008-2009
% \item Member -- SIAM
% \end{innerlist}

% \section{Expertise}
% %
% Mathematics:
% %
% \begin{innerlist}
%     \item Applied Mathematics, Real and Complex Analysis, Measure
%         Theory, Differential Geometry, Game Theory, Graph Theory,
%         Combinatorics
% \end{innerlist}
%
% \halfblankline
%
% Control Theory and Engineering:
% %
% \begin{innerlist}
%     \item Linear and Nonlinear Systems Theory, Feedback, Variable
%         Structure Systems and Sliding Modes, Distributed and Intelligent
%         Control, Dynamic Optimization, Bio-mimicry, Bio-inspiration,
%         Hybrid and CyberPhysical Systems
% \end{innerlist}
%
% \halfblankline
%
% Communications and Signal Processing:
% %
% \begin{innerlist}
%     \item Probability, Random Variables, Stochastic Processes,
%         Information Theory, Estimation, Networks
% \end{innerlist}
%
% \halfblankline
%
% Computer Science and Engineering:
% %
% \begin{innerlist}
%     \item Model Checking (automated, distributed, hybrid,
%         probabilistic), Hybrid Automata, Software Verification,
%         Component-Based Reusable Software
% \end{innerlist}
%
% \halfblankline
%
% Natural Sciences (Biology, Neuroscience, Psychology, Anthropology):
% %
% \begin{innerlist}
%     \item Behavioral Ecology, Foraging Theory, Altruism, Impulsiveness,
%         Evolution
% \end{innerlist}
%
% \section{References Available to Contact}
% %
% \href
% {http://www.ece.osu.edu/~passino/}
% {\textbf{Dr.~Kevin M.~Passino}}
% (e-mail:~\href{mailto:passino.1@osu.edu}{passino.1@osu.edu}; phone:~+1-614-312-2472)
% %
% \begin{innerlist}
%     \item Professor,
%         \href{http://www.ece.osu.edu/}{Electrical and Computer
%         Engineering},
%         \href{http://www.osu.edu/}{The Ohio State University}
%
%     \item[$\diamond$] 205 Dreese Laboratories, 2015 Neil Ave., Columbus,
%         OH  43210
%
%     \item[$\star$] \emph{Dr.~Passino was my graduate adviser.}
% \end{innerlist}
%
% \halfblankline
%
% \href
% {http://www.cse.ohio-state.edu/~weide/}
% {\textbf{Dr.~Bruce W.~Weide}}
% (e-mail:~\href{mailto:weide.1@osu.edu}{weide.1@osu.edu}; phone:~+1-614-292-1517)
% \begin{innerlist}
%     \item Professor and Associate Chair,
%         \href{http://www.cse.ohio-state.edu/}{Computer Science and
%         Engineering}\\
%         \href{http://www.osu.edu/}{The Ohio State University}
%
%     \item[$\diamond$] 395 Dreese Laboratories, 2015 Neil Ave., Columbus,
%         OH  43210
%
%     \item[$\star$] \emph{Dr.~Weide is a co-PI on the NSF grant that
%         funds my current postdoctoral position.}
% \end{innerlist}
%
% \halfblankline
%
% \href
% {http://hamilton-lab.wikidot.com/}
% {\textbf{Dr.~Ian M.~Hamilton}}
% (e-mail:~\href{mailto:hamilton.598@osu.edu}{hamilton.598@osu.edu}; phone:~+1-614-292-9147)
% %
% \begin{innerlist}
%     \item Assistant Professor,
%         \href{http://eeob.osu.edu/}{Evolution, Ecology, and Organismal Biology}
%         and
%         \href{http://www.math.ohio-state.edu/}{Mathematics}\\
%         \href{http://www.osu.edu/}{The Ohio State University}
%
%     \item[$\diamond$] 300 Aronoff Laboratory, 318 W.~12th Avenue,
%         Columbus, OH  43210
%
%     \item[$\star$] \emph{Dr.~Hamilton has been a valuable
%         interdisciplinary resource to me.}
% \end{innerlist}
%
% \halfblankline
%
% \href
% {http://www.ece.osu.edu/~serrani/}
% {\textbf{Dr.~Andrea Serrani}}
% (e-mail:~\href{mailto:serrani.1@osu.edu}{serrani.1@osu.edu}; phone:~+1-614-292-4976)
% %
% \begin{innerlist}
%     \item Associate Professor,
%         \href{http://www.ece.osu.edu/}{Electrical and Computer Engineering}\\
%         \href{http://www.osu.edu/}{The Ohio State University}
%
%     \item[$\diamond$] 205 Dreese Laboratories, 2015 Neil Ave., Columbus,
%         OH  43210
%
%     \item[$\star$] \emph{Dr.~Serrani was a member of my doctoral
%         committee.}
% \end{innerlist}
%
% \halfblankline
%
% \href
% {http://www.cse.ohio-state.edu/~paolo/}
% {\textbf{Dr.~Paolo A.~G.~Sivilotti}}
% (e-mail:~\href{mailto:sivilotti.1@osu.edu}{sivilotti.1@osu.edu}; phone: +1-614-292-5835)
% \begin{innerlist}
%     \item Associate Professor,
%         \href{http://www.cse.ohio-state.edu/}{Computer Science and Engineering},
%         \href{http://www.osu.edu/}{The Ohio State University}
%
%     \item[$\diamond$] 395 Dreese Laboratories, 2015 Neil Ave., Columbus,
%         OH  43210
%
%     \item[$\star$] \emph{Dr.~Sivilotti is a co-PI on the NSF grant that
%         funds my current postdoctoral position.}
% \end{innerlist}
%
% \halfblankline
%
% \href
% {http://feh.osu.edu/staff/view.html?UID=798}
% {\textbf{Dr.~Richard J.~Freuler}}
% (e-mail:~\href{mailto:freuler.1@osu.edu}{freuler.1@osu.edu}; phone: +1-614-688-0499)
% \begin{innerlist}
%     \item Professor of Practice,
%         \href{http://mae.osu.edu/}{Mechanical and Aerospace Engineering}\\
%         \href{http://www.osu.edu/}{The Ohio State University}
%
%     \item[$\diamond$] 244 Hitchcock Hall, 2070 Neil Ave., Columbus, OH  43210
%
%     \item[$\star$] \emph{Dr.~Freuler coordinates the Fundamentals of
%         Engineering for Honors program in which I served as an
%         instructor early in my academic career.}
% \end{innerlist}
%
% \halfblankline
%
% \href
% {http://mae.osu.edu/people/staab.1}
% {\textbf{Dr.~George H.~Staab}}
% (e-mail:~\href{mailto:staab.1@osu.edu}{staab.1@osu.edu}; phone: +1-614-292-7920)
% \begin{innerlist}
%     \item Associate Professor,
%         \href{http://mae.osu.edu/}{Mechanical and Aerospace Engineering}\\
%         \href{http://www.osu.edu/}{The Ohio State University}
%
%     \item[$\diamond$] W192 Scott Laboratory, 201 W.~19th Ave., Columbus, OH  43210
%
%     \item[$\star$] \emph{Dr.~Staab is the faculty adviser for the OSU
%         FIRST robotics and engineering outreach group of which I was a
%         four-year member and team leader.}
% \end{innerlist}
%
% \halfblankline
%
% \textbf{Dr.~Clayton Daigle}
% (e-mail:~\href{mailto:Clayton.Daigle@silabs.com}{Clayton.Daigle@silabs.com}; phone: +1-512-532-5935)
% \begin{innerlist}
%     \item Mixed-Signal Engineer,
%         \href{http://www.silabs.com/}{Silicon Laboratories}, Austin, TX
%
%     \item[$\star$] \emph{Dr.~Daigle was my direct supervisor when I
%         worked for National Instruments as an analog hardware R\&D
%         engineer.}
% \end{innerlist}
%
\end{document}

%%%%%%%%%%%%%%%%%%%%%%%%%% End CV Document %%%%%%%%%%%%%%%%%%%%%%%%%%%%%

%----------------------------------------------------------------------%
% The following is copyright and licensing information for
% redistribution of this LaTeX source code; it also includes a liability
% statement. If this source code is not being redistributed to others,
% it may be omitted. It has no effect on the function of the above code.
%----------------------------------------------------------------------%
% Copyright (c) 2007, 2008, 2009, 2010, 2011 by Theodore P. Pavlic
%
% Unless otherwise expressly stated, this work is licensed under the
% Creative Commons Attribution-Noncommercial 3.0 United States License. To
% view a copy of this license, visit
% http://creativecommons.org/licenses/by-nc/3.0/us/ or send a letter to
% Creative Commons, 171 Second Street, Suite 300, San Francisco,
% California, 94105, USA.
%
% THE SOFTWARE IS PROVIDED "AS IS", WITHOUT WARRANTY OF ANY KIND, EXPRESS
% OR IMPLIED, INCLUDING BUT NOT LIMITED TO THE WARRANTIES OF
% MERCHANTABILITY, FITNESS FOR A PARTICULAR PURPOSE AND NONINFRINGEMENT.
% IN NO EVENT SHALL THE AUTHORS OR COPYRIGHT HOLDERS BE LIABLE FOR ANY
% CLAIM, DAMAGES OR OTHER LIABILITY, WHETHER IN AN ACTION OF CONTRACT,
% TORT OR OTHERWISE, ARISING FROM, OUT OF OR IN CONNECTION WITH THE
% SOFTWARE OR THE USE OR OTHER DEALINGS IN THE SOFTWARE.
%----------------------------------------------------------------------%
